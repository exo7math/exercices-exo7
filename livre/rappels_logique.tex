\documentclass[11pt,class=report,crop=false]{standalone}
\usepackage[screen]{exo7book}

\begin{document}

%====================================================================
\chapitre[Logique et raisonnements]{Logique et\\ raisonnements}
%====================================================================

\insertvideo{aWSe1fjJHEM}{partie 1. Logique}

\insertvideo{B-I5yZd0Wbk}{partie 2. Raisonnements}

\insertfiche{fic00002.pdf}{Logique, ensembles, raisonnements}


%%%%%%%%%%%%%%%%%%%%%%%%%%%%%%%%%%%%%%%%%%%%%%%%%%%%%%%%%%%%%%%%
\section{Logique}
\index{logique}

%---------------------------------------------------------------
\subsection{Assertions}

Une \defi{assertion}\index{assertion} est une phrase soit vraie, soit fausse, pas les deux en même temps. 
Exemples : \assertion{$2\times 3 = 7$} est une assertion fausse ;  \assertion{Pour tout $x \in \Rr$, on a $x^2 \ge 0$} est une assertion vraie.


%--------------
\subsubsection*{Les opérateur logiques}

\begin{itemize}
  \item L'assertion \assertion{$P$ \defi{et} $Q$}\index{et logique@\og et \fg{} logique} est vraie si $P$ est vraie et $Q$ est vraie. L'assertion \assertion{$P$ et $Q$} est fausse sinon.

  \item L'assertion \assertion{$P$ \defi{ou} $Q$}\index{ou logique@\og ou \fg{} logique} est vraie si l'une (au moins) des deux assertions $P$ ou $Q$ est vraie.
L'assertion \assertion{$P$ ou $Q$} est fausse si les deux assertions $P$ et $Q$ sont fausses.

  \item L'assertion \assertion{\defi{non} $P$}\index{négation} est vraie si $P$ est fausse, et fausse si $P$ est vraie.
\end{itemize}

On résume ceci dans les \defi{tables de vérité}\index{table de verite@table de vérité} :


\begin{center}
\begin{tabular}{ccc}
\begin{tabular}{c|c|c}
\textcolor{olive}{$P$} $\backslash$ \textcolor{blue}{$Q$} & \textcolor{blue}{V} & \textcolor{blue}{F} \\ \hline
\textcolor{olive}{V} & \textcolor{red}{V} & \textcolor{red}{F} \\ \hline
\textcolor{olive}{F} & \textcolor{red}{F} & \textcolor{red}{F} \\
\end{tabular}
&
\begin{tabular}{c|c|c}
\textcolor{olive}{$P$} $\backslash$ \textcolor{blue}{$Q$}  & \textcolor{blue}{V} & \textcolor{blue}{F} \\ \hline
\textcolor{olive}{V} & \textcolor{red}{V} & \textcolor{red}{V} \\ \hline
\textcolor{olive}{F} & \textcolor{red}{V} & \textcolor{red}{F} \\
\end{tabular}
&
\begin{tabular}{c|c|c}
 \textcolor{blue}{$P$}  &  \textcolor{blue}{V} &  \textcolor{blue}{F} \\ \hline
 \textcolor{red}{non $P$}    & \textcolor{red}{F} & \textcolor{red}{V} \\
\end{tabular}
\\
{Table de vérité de \assertion{$P$ et $Q$}}
&{Table de vérité de \assertion{$P$ ou $Q$}}
&{Table de vérité de \assertion{non $P$}}
\end{tabular}
\end{center}


%--------------
\subsubsection*{L'implication}

\mybox{
L'assertion \assertion{(non $P$) ou $Q$} est notée \assertion{$P \implies Q$}\index{implication}\index{$\implies$}.
}
L'assertion \assertion{$P \implies Q$} se lit en français \assertion{$P$ implique $Q$}.
Sa table de vérité est donc :
\begin{center}
\begin{tabular}{c|c|c}
\textcolor{olive}{$P$} $\backslash$ \textcolor{blue}{$Q$}  & \textcolor{blue}{V} & \textcolor{blue}{F} \\ \hline
\textcolor{olive}{V} & \textcolor{red}{V} & \textcolor{red}{F} \\ \hline
\textcolor{olive}{F} & \textcolor{red}{V} & \textcolor{red}{V} \\
\end{tabular}
\end{center}


Exemple :
\begin{itemize}
  \item \assertion{$0 \le x \le 25 \implies \sqrt x \le 5$}  est vraie (prendre la racine carrée).
  \item \assertion{$2+2=5 \implies \sqrt 2 = 2$}  est vraie ! Eh oui, si
$P$ est fausse alors l'assertion \assertion{$P \implies Q$} est toujours vraie.
\end{itemize}



%--------------
\subsubsection*{L'équivalence}


L'\defi{équivalence} est définie par :
\mybox{
\assertion{$P \iff Q$}\index{equivalence@équivalence}\index{$\iff$}  est l'assertion \assertion{($P \implies Q$) \  et \  ($Q \implies P$)}.
}

\begin{proposition}
\sauteligne
\begin{itemize}
  \item \emph{$\text{non}(P \text{ et } Q)  \iff  (\text{non } P)  \text{ ou } (\text{non }Q)$}
  \item \emph{$\text{non}(P \text{ ou } Q)  \iff  (\text{non } P)  \text{ et } (\text{non }Q)$}
  \item \emph{$\big(P \text{ et } (Q \text{ ou } R)  \big)   \iff
(P \text{ et } Q) \text{ ou } (P \text{ et }  R)$}
  \item \emph{$\big(P \text{ ou } (Q \text{ et } R)  \big)   \iff
(P \text{ ou } Q) \text{ et } (P \text{ ou }  R)$}
\end{itemize}
\end{proposition}



%---------------------------------------------------------------
\subsection{Quantificateurs}
\index{quantificateur}

%--------------
\subsubsection*{Le quantificateur $\forall$ : \assertion{pour tout}}
\index{$\forall$}


L'assertion \assertion{$\forall x \in E \quad P(x)$}
est une assertion vraie lorsque les assertions $P(x)$ sont vraies pour tous les éléments $x$
de l'ensemble $E$.


Par exemple : \assertion{$\forall x \in \Rr \quad (x^2\ge 1)$}  est une assertion fausse.



%--------------
\subsubsection*{Le quantificateur $\exists$ : \assertion{il existe}}
\index{$\exists$}

L'assertion \assertion{$\exists x \in E \quad P(x)$}
est une assertion vraie lorsque l'on peut trouver au moins un $x$ de $E$ pour lequel $P(x)$ est vraie.

Exemple : \assertion{$\exists n \in \Nn \quad n^2-n > n$}  est vraie (il y a plein de choix, par exemple $n=3$ convient, mais aussi $n=10$ ou même $n=100$, un seul suffit pour dire que l'assertion est vraie).
 


%--------------
\subsubsection*{La négation des quantificateurs}

\mybox{
La négation de \assertion{$\forall x \in E \quad P(x)$} \ \ est \ \
\assertion{$\exists x \in E \quad \text{non } P(x)$} .
}

Par exemple la négation de \assertion{$\forall x \in [1,+\infty[ \quad (x^2\ge 1)$}
est l'assertion \assertion{$\exists x \in [1,+\infty[ \quad (x^2 < 1)$}.
En effet la négation de $x^2\ge 1$ est $\text{non}(x^2 \ge 1)$ mais s'écrit plus simplement $x^2 < 1$.


\mybox{
La négation de \assertion{$\exists x \in E \quad P(x)$} \ \ est \ \ \assertion{$\forall x \in E \quad \text{non } P(x)$}.
}

Exemple : pour la phrase 
\assertion{$\forall x \in \Rr \quad \exists y >0 \quad (x+y > 10)$}
sa négation est
\assertion{$\exists x \in \Rr \quad \forall y > 0 \quad (x+y \le 10).$}

%--------------
\subsubsection*{Ordre des quantificateurs}

L'ordre des quantificateurs est très important.
Exemple : les deux phrases logiques suivantes sont différentes (la première est vraie, la seconde est fausse) :
 $$\forall x \in \Rr \quad  \exists y \in \Rr \quad (x+y>0) \qquad \text{et}\qquad
\exists y \in \Rr \quad \forall x \in \Rr \quad (x+y>0).$$
 







%%%%%%%%%%%%%%%%%%%%%%%%%%%%%%%%%%%%%%%%%%%%%%%%%%%%%%%%%%%%%%%%
\section{Raisonnements}
\index{raisonnement}

%---------------------------------------------------------------
\subsection{Raisonnement direct}
On veut montrer que l'assertion \assertion{$P \implies Q$} est vraie.
On suppose que $P$ est vraie et on montre qu'alors $Q$ est vraie.
C'est la méthode à laquelle vous êtes le plus habitué.



%---------------------------------------------------------------
\subsection{Cas par cas}

Si l'on souhaite vérifier une assertion $P(x)$ pour tous les $x$ dans un ensemble $E$, on
montre l'assertion pour les $x$ dans une partie $A$ de $E$, puis pour les $x$
n'appartenant pas à $A$. C'est la méthode de \defi{disjonction}\index{disjonction} ou du \defi{cas par cas}.



%---------------------------------------------------------------
\subsection{Contraposée}

Le raisonnement par \defi{contraposition}\index{contraposition} est basé sur l'équivalence suivante :
\mybox{
L'assertion \assertion{$P \implies Q$}  \  est équivalente à \  \assertion{$\text{non}(Q) \implies \text{non}(P)$}.
}

Donc si l'on souhaite montrer l'assertion \assertion{$P \implies Q$}, on montre en fait
que si $\text{non}(Q)$ est vraie alors $\text{non}(P)$ est vraie.




%---------------------------------------------------------------
\subsection{Absurde}

Le \defi{raisonnement par l'absurde}\index{absurde} pour montrer \assertion{$P \implies Q$}  repose sur le principe suivant :
on suppose à la fois que $P$ est vraie et que $Q$ est fausse et on cherche une contradiction.
Ainsi si $P$ est vraie alors $Q$ doit être vraie et donc \assertion{$P \implies Q$} est vraie.



%---------------------------------------------------------------
\subsection{Contre-exemple}

Si l'on veut montrer qu'une assertion du type \assertion{$\forall x \in E \quad P(x)$} est vraie
alors pour chaque $x$ de $E$ il faut montrer que $P(x)$ est vraie. Par contre
pour montrer que cette assertion est fausse alors il suffit de trouver $x \in E$ tel que $P(x)$
 soit fausse. (Rappelez-vous la négation de \assertion{$\forall x \in E \quad P(x)$}
est \assertion{$\exists x \in E \quad \text{non}\ P(x)$}.)
Trouver un tel $x$ c'est trouver un \defi{contre-exemple}\index{contre-exemple} à l'assertion \assertion{$\forall x \in E \quad P(x)$}.



%---------------------------------------------------------------
\subsection{Récurrence}

Le \defi{principe de récurrence}\index{recurrence@récurrence} permet de montrer qu'une assertion $P(n)$, dépendant de $n$,
est vraie pour tout $n \in \Nn$.
La démonstration par récurrence se déroule en trois étapes : lors de l'\evidence{initialisation} on prouve $P(0)$.
Pour l'étape d'\evidence{hérédité}\index{heredite@hérédité}, on suppose $n\ge 0$ donné avec $P(n)$ vraie,
et on démontre alors que l'assertion  $P(n+1)$ au rang suivant est vraie.
Enfin dans la \evidence{conclusion}, on rappelle que par le principe de récurrence $P(n)$ est vraie pour tout $n\in\Nn$.


\finchapitre
\end{document}


