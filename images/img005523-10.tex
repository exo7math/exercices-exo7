%%%%%%%%%%%%%%%%%% PREAMBULE %%%%%%%%%%%%%%%%%%

\documentclass[12pt,a4paper]{article}

\usepackage{amsfonts,amsmath,amssymb,amsthm}
\usepackage[francais]{babel}
\usepackage[utf8]{inputenc}
\usepackage[T1]{fontenc}

%----- Ensemles : entiers, reels, complexes -----
\newcommand{\Nn}{\mathbb{N}} \newcommand{\N}{\mathbb{N}}
\newcommand{\Zz}{\mathbb{Z}} \newcommand{\Z}{\mathbb{Z}}
\newcommand{\Qq}{\mathbb{Q}} \newcommand{\Q}{\mathbb{Q}}
\newcommand{\Rr}{\mathbb{R}} \newcommand{\R}{\mathbb{R}}
\newcommand{\Cc}{\mathbb{C}} \newcommand{\C}{\mathbb{C}}

%----- Modifications de symboles -----
\renewcommand {\epsilon}{\varepsilon}
\renewcommand {\Re}{\mathop{\mathrm{Re}}\nolimits}
\renewcommand {\Im}{\mathop{\mathrm{Im}}\nolimits}

%----- Fonctions usuelles -----
\newcommand{\ch}{\mathop{\mathrm{ch}}\nolimits}
\newcommand{\sh}{\mathop{\mathrm{sh}}\nolimits}
\renewcommand{\tanh}{\mathop{\mathrm{th}}\nolimits}
\newcommand{\Arcsin}{\mathop{\mathrm{Arcsin}}\nolimits}
\newcommand{\Arccos}{\mathop{\mathrm{Arccos}}\nolimits}
\newcommand{\Arctan}{\mathop{\mathrm{Arctan}}\nolimits}
\newcommand{\Argsh}{\mathop{\mathrm{Argsh}}\nolimits}
\newcommand{\Argch}{\mathop{\mathrm{Argch}}\nolimits}
\newcommand{\Argth}{\mathop{\mathrm{Argth}}\nolimits}
\newcommand{\pgcd}{\mathop{\mathrm{pgcd}}\nolimits} 

%----- Commandes special dessin a ajouter localement ------
\usepackage{geometry}
\usepackage{pstricks}
\usepackage{pst-plot}
\usepackage{pst-node}
\usepackage{graphics,epsfig}

\pagestyle{empty}

% Que faire avec ce fichier monimage.tex ?
%   1/ latex monimage.tex
%   2/ dvips monimage.dvi
%   3/ ps2eps monimage.ps
%   4/ ps2pdf -dEPSCrop monimage.eps
%   5/ Dans votre fichier d'exos \includegraphics{monimage}

\begin{document}

\begin{pspicture}(-8.1,-4.2)(0.1,4.2)
\psset{xunit=3cm,yunit=3cm}
\psaxes{->}(0,0)(-1.3,-1.3)(1.5,1.3)
\parametricplot[linecolor=blue,plotpoints=100]{0}{360}{t dup 2 mul sin exch 3 mul sin}
\psline[linestyle=dashed](-1,-1)(-1,1)
\psline[linestyle=dashed](-1,1)(1,1)
\psline[linestyle=dashed](1,1)(1,-1)
\psline[linestyle=dashed](1,-1)(-1,-1)
\uput[ur](0.1,0){$t=0$}
\uput[u](0.866,1){$t=\pi/6$}
\uput[r](1,0.707){$t=\pi/4$}
\uput[ur](0.866,0){$t=\pi/3$}
\uput[d](0,-1){$t=\pi/2$}
\psline[linecolor=red]{<->}(0.66,1)(1.06,1)
\psline[linecolor=red]{<->}(-0.66,1)(-1.06,1)
\psline[linecolor=red]{<->}(-0.66,-1)(-1.06,-1)
\psline[linecolor=red]{<->}(0.66,-1)(1.06,-1)
\psline[linecolor=red]{<->}(1,0.907)(1,0.507)
\psline[linecolor=red]{<->}(-1,0.907)(-1,0.507)
\psline[linecolor=red]{<->}(1,-0.907)(1,-0.507)
\psline[linecolor=red]{<->}(-1,-0.907)(-1,-0.507)
\psline[linecolor=red]{<->}(-0.2,1)(0.2,1)
\psline[linecolor=red]{<->}(-0.2,-1)(0.2,-1)
\psdots[linecolor=blue](0,0)(0.86,1)(1,0.707)(0.86,0)(0,-1)(0,1)(-0.86,1)(-1,0.707)(-0.86,0)(-0.86,-1)(-1,-0.707)
\end{pspicture}

\end{document}
